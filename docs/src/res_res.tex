\begin{flushleft}
    For each of the following dataset results, my solution was run over 50 independent runs for priori number of \\
    clusters and 50 independent runs for each possible optimal number of clusters in the range from 2 to 15. \\
    Therefore, the results showed below, are averages of those 50 independent runs.\\
    \subsection{Iris Dataset Results}
        \subsubsection{Priori Number of Clusters}
            \begin{itemize}
                \item Number of clusters: 3
                \item Average number of samples per cluster: {50, 52, 48}
            \end{itemize}
            Given that the actual number of samples per cluster is {50, 50, 50}, \\
            my implementation has an error rate of 2.667% (i.e. 4/150 samples were incorrectly clustered). \\ 
        \subsubsection{Dynamic Optimal Number of Clusters}
            \begin{itemize}
                \item Determined optimal number of clusters: 3
                \item Average number of samples per cluster: {53, 50, 47}
            \end{itemize}
            Given that the actual number of cluster is 3, my implementation was able to determine the correct optimal \\
            number of clusters each time using the approach mentioned in the Implementation section. \\
            For that optimal number of clusters, the average number of samples per cluster over those 50 independent \\
            runs was {54, 50, 47}. Given that the actual number of samples per cluster is {50, 50, 50}, \\
            my implementation has an error rate of 4% (i.e. 6/150 samples were incorrectly clustered). \\

    \subsection{Wine Dataset Results}
        \subsubsection{Priori Number of Clusters}
            \begin{itemize}
                \item Number of clusters: 3
                \item Average number of samples per cluster: {73, 71, 35}
            \end{itemize}
            Given that the actual number of samples per cluster is {59, 71, 48}, \\
            my implementation has an error rate of 15.084% (i.e. 27/179 samples were incorrectly clustered). \\ 
        \subsubsection{Dynamic Optimal Number of Clusters}
            \begin{itemize}
                \item Determined optimal number of clusters: 4
                \item Average number of samples per cluster: {46, 44, 46, 40}
            \end{itemize}
            Given that the actual number of cluster is 3, my implementation was off by 1 \\
            cluster each time using the approach mentioned in the Implementation section. This is due to \\
            the fact that the dataset was not normalized and thus skewed the results in favour of the attribute \\
            with the highest value. To try and mitigate this effect, one could preprocess the dataset such that all \\
            attributes are on the same scale, i.e. from 0 to 1. Also, one could apply Principal Component Analysis \\
            (PCA) reduction on the dataset such that only "relevant" features of the dataset are utilized.
            For that optimal number of clusters, the average number of samples per cluster over those 50 independent \\
            runs was {46, 44, 46, 40}. Given that the actual number of samples per cluster is {59, 71, 48}, \\
            my implementation has an error rate of 45.810% (i.e. 83/179 samples were incorrectly clustered). \\

    \subsection{Epileptic Seizure Recognition Dataset Results}
        \subsubsection{Priori Number of Clusters}
            \begin{itemize}
                \item Number of clusters: 5
                \item Average number of samples per cluster: {3267, 2202, 3204, 2668, 159}
            \end{itemize}
            Given that the actual number of samples per cluster is {2300, 2300, 2300, 2300, 2300}, \\
            my implementation has an error rate of 38.939% (i.e. 4478/11500 samples were incorrectly clustered). \\
        \subsubsection{Dynamic Optimal Number of Clusters}
            \begin{itemize}
                \item Determined optimal number of clusters: 12
                \item Average number of samples per cluster: {1298, 800, 977, 982, 666, 395, 1198, 934, 736, 1190, 1253, 1066}
            \end{itemize}
            Given that the actual number of samples per cluster is {2300, 2300, 2300, 2300, 2300}, \\
            my implementation has an error rate of 47.783% (i.e. 5495/11500 samples were incorrectly clustered). \\
\end{flushleft}