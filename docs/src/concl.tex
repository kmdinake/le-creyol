\begin{flushleft}
    \subsection{Solution Feasibility}
        The focus of this assignment was to determine whether a MOPSO could be used to solve a static \\
        data clustering problem, and In this report I have shown a method of how it could be achieved. \\
    \subsection{Performance Observations}
        The research results in the section above show that on different classes of data with varying \\
        dimensions, the algorithm was able to produce some fairly accurate results.\\
        On low dimensions the results were near 100% accuracy each attempt at executing the solution, as \\
        in the case of the Iris dataset results. However, on the Wine dataset where the number of data \\
        samples were roughly the same, the difference factor came in on the dimensions of the data. \\
        Wine has 13 dimensions per data sample whereas Iris has only 4, and I believe that the difference \\
        in the performance on the results is largely due to the varying scale of the features and the \\
        chance that we're considering "non-relevant" features. So PCA reduction could be utilized, among \\
        other dimension reduction algorithms. This issue of dimensionality can also be seen with the performance \\
        of the solution on the Epileptic Seizure Recognition dataset, which had a dimensionality of 6 and \\
        over 11000 data samples. It is worth noting that on 2/3 datasets, the optimal number of \\
        clusters was mis-classified, and thus the performance of the data clustering was slightly better than just \\
        randomly assignment of a data sample to a cluster. My understanding leads me to believe that Particle Swarm \\
        Optimization may not perform well with high dimensional data. \\
        Note that the above observations were consistent whether dynamic number of clusters were to be determined \\
        or whether they were known prior to execution. \\
    \subsection{Final words}
    In my humble opinion, this is probably not the best use case for MOPSO, considering that the training takes ages \\
    to complete. Also, it is worth mentioning that my approach has several flaws as pointed above and could possible be \\
    mitigated through choosing a different method of clustering, add data preprocessing techniques, and use several PSO \\
    performance enhancement methods to ensure better results. \\
    With that being said, I enjoyed getting to work on such an assignment and look foward to learning more about Swarm Intelligence.\\
\end{flushleft}